\documentclass[]{article}
\usepackage[hidelinks]{hyperref}
\usepackage{amsmath}
\usepackage{amsthm}
\usepackage{amssymb}
\theoremstyle{definition}
\newtheorem{definition}{Definition}[section]
\begin{document}
	
\section{Complexity Zoo}

\subsection{TIME[f(n)]}
\label{sec:TIME}
Informally: problems that can be solved in f(n) time. 
\begin{definition}
	Given some function $f : \mathbb{N} \to \mathbb{N}$, $\text{TIME}[f(n)]$ are the set of problems solvable within $O(f(n))$ atomic steps on a deterministic Turing machine. Where $n$ is the size of the input.
\end{definition}

\subsection{NTIME[f(n)]}
\label{sec:NTIME}
Informally: problems that can be solved nondeterministically in f(n) time. 
\begin{definition}
	Given some function $f : \mathbb{N} \to \mathbb{N}$, $\text{NTIME}[f(n)]$ are the set of problems solvable within $O(f(n))$ atomic steps on a nondeterministic Turing machine.
\end{definition}

\subsection{SPACE[f(n)]}
\label{sec:SPACE}
Informally: problems that can be solved in f(n) space. 
\begin{definition}
	Given some function $f : \mathbb{N} \to \mathbb{N}$, $\text{SPACE}[f(n)]$ are the set of problems solvable using a tape of length $O(f(n))$ on a deterministic Turing machine. Where $n$ is the size of the input.
\end{definition}

\subsection{NSPACE[f(n)]}
\label{sec:NSPACE}
Informally: problems that can be solved non-deterministically in f(n) space. 
\begin{definition}
	Given some function $f : \mathbb{N} \to \mathbb{N}$, $\text{NSPACE}[f(n)]$ are the set of problems solvable using a tape of length $O(f(n))$ on a non-deterministic Turing machine. Where $n$ is the size of the input.
\end{definition}


\subsection{$\mathbf{P}$}
\label{sec:P}
Informally: all problems that can be solved in polynomial time.
\begin{definition}
	$$\mathbf{P} = \bigcup_{k\geq 0} \hyperref[sec:TIME]{\text{TIME}}[n^{k}]$$
\end{definition}
$ $
\\
\\
Descriptive Complexity definitions: 
\begin{definition}
	$$\mathbf{P} = \text{FO(LFP)}$$
(First Order logic extended with the Least Fixed Point operator, with successor. A high level, handwavy description of the LFP operator is the added ability to recursively define FO formulas.)
\end{definition}
\begin{definition}
	$$\mathbf{P} = \text{SO(Horn)}$$
(Second Order logic restricted with Horn. SO logic allows you to quantify over subsets/relations/functions on the domain, and Horn means all `clauses' are really implications with literal in the conclusion and all literals positive.)
\end{definition}
$ $
\\
\\
Circuit Complexity definition:
\begin{definition}
	$$\mathbf{P} = \text{Set of problems that can be solved by a polynomial-time uniform family of boolean circuits}$$
\end{definition}
$ $
\\
\\
Notable Problems in $\mathbf{P}$:
\begin{itemize}
	\item 2-SAT
	\item 2-Colourability
	\item Reachability
\end{itemize}
\subsection{$\mathbf{NP}$}
\label{sec:NP}
Informally: all problems that can be solved in nondeterministic polynomial time.
\begin{definition}
	$$\mathbf{NP} = \bigcup_{k\geq 0} \hyperref[sec:NTIME]{\text{NTIME}}[n^{k}]$$
\end{definition}
$ $
\\
\\
In terms of a verifier:
\\
Informally: The set of decision problems where a solution can be verified in polynomial time.
\\
\\
Descriptive Complexity Definition:
\begin{definition}
	$$\mathbf{NP} = \text{SO}\exists$$
	(Existential Second Order)
\end{definition}
$ $
\\
\\
Notable Problems in $\mathbf{NP}$:
\begin{itemize}
	\item SAT
	\item 3-Colourability
	\item TSP
	\item Subset sum
\end{itemize}
\subsection{FPT}
\label{sec:FPT}
Informally, the set of problems that can be solved in polynomial time for some fixed parameter.
\\
\begin{definition}
	The set of problems that can be parameterised by $k$ and can be solved in $f(k)n^c$, where $f(x)$ is only dependent on $k$, and $c$ is an independent constant.
\end{definition}
$ $
\\
\hyperref[sec:P]{$\mathbf{P}$} is contained within $\mathbf{FPT}$.
\\
If a problem is in $\mathbf{FPT}$, then for any fixed $k$ that problem is in \hyperref[sec:P]{$\mathbf{P}$}.
\\
$\mathbf{FPT}$ is also known as \hyperref[sec:W]{$\mathbf{W[0]}$}
$ $
\\
\\
Notable Problems in $\mathbf{FPT}$:
\begin{itemize}
	\item Vertex Cover
\end{itemize}
\subsection{W[1]}
\label{sec:W[1]}

\begin{definition}
	The class of parametrized problems that admit a parametrized reduction to the following problem:
	Given a	nondeterministic single-tape Turing machine, decide if it accepts within k steps.
\end{definition}
N.B This is short acceptance
\begin{definition}
	The class of parametrized problems that admit a parametrized reduction to the following problem:
	Given a Boolean circuit C, with a mixture of fanin-2 and unbounded-fanin gates. There is at most 1 unbounded-fanin gate along any path to the root, and the total depth (fanin-2 and unbounded-fanin) is constant. Does C have a satisfying assignment of Hamming weight k?
\end{definition}
N.B This is Weighted 3-SAT.
\\
\\
Notable Problems in $\mathbf{W[1]}$:
\begin{itemize}
	\item Short Acceptance
	\item Weighted 3-SAT
	\item Clique (of size k)
	\item Independent set (of size k)
\end{itemize}
\subsection{W[2]}
\subsection{W[i]}
\label{sec:W}
\subsection{FPTAS}

\subsection{PTAS}

\subsection{L}
Informally: all problems that can be solved using logarithmic space (excluding the input)
\\
\begin{definition}
	$\mathbf{L} = \hyperref[sec:SPACE]{\text{SPACE}[\log n]}$
\end{definition}
$ $
\\
This means you effectively have the input and then a fixed number of counters/pointers (up to the size of the input)
\\
\\
Notable Problems in $\mathbf{L}$:
\begin{itemize}
	\item Planar Graph Isomorphism
\end{itemize}
\subsection{NL}
Informally: all problems that can be solved using nondeterministic logarithmic space (excluding the input)
\\
\begin{definition}
	$\mathbf{NL} = \hyperref[sec:NSPACE]{\text{NSPACE}[\log n]}$
\end{definition}
$ $
\\
This means you effectively have the input and then a fixed number of counters/pointers (up to the size of the input)
\\
\begin{definition}
	$\mathbf{NL} = \mathbf{coNL}$
\end{definition}
$ $
\\
Notable Problems in $\mathbf{NL}$:
\begin{itemize}
	\item Reachability
	\item Unreachability
\end{itemize}

\subsection{PSPACE}

\subsection{coNP}

\subsection{$\Sigma^p_2$}

\subsection{$\Sigma^p_i$}

\subsection{$\Pi^p_2$}

\subsection{$\Pi^p_i$}

\subsection{PH}

\subsection{$P^{SAT}$}

\subsection{$NP^{SAT}$}

\subsection{P/poly}

\subsection{P-Uniform}

\subsection{EXP}

\subsection{NC}

\subsection{$NC_0$}

\subsection{$NC_1$}

\subsection{$NC_2$}

\subsection{$NC_i$}

\subsection{$AC_i$}

\subsection{$AC_0$}

\subsection{$AC_1$}

\subsection{BPP}

\subsection{RP}

\subsection{co-RP}

\subsection{ZPP}

\subsection{APX}

\subsection{PO}

\subsection{PCP}

\subsection{BQP}

\subsection{$\# P$}

\subsection{PPAD}

\subsection{co-NP}
\end{document}
